\documentclass[../report.tex]{subfiles}
\begin{document}
\graphicspath{{img/}{../img/}}

\paragraph{Predicate serialization}
As mentioned in the implementation chapter, an obstacle occurred with serialization of predicates. This problem limited the usage of Ocon, but it was not possible to include the feature in this project. A framework for serializing the predicates as expression trees should be doable, but that is left for a later iteration of Ocon.


\paragraph{Performance}
%While performance is not in scope it is still worth a few remarks, since it can have a big impact with our implementation approach. In worst case we have an $ O(situations\times entities) $ growth in the OconContextFilter when checking predicates, since for each situation every entity might be looped over. However, as the predicate delegate is assigned by an implementor of Ocon it is up to him what he does with the entity collection. Therefore in best case the growth is $ O(situations) $. Performance could be improved upon by implementing an algorithm to only check relevant entity types for a situation.

While performance is not in scope it is still worth a few remarks, since it can have a big impact with our implementation approach. The predicate delegate is assigned by an implementor of Ocon and it is up to him what he does with the entity collection. Therefore in best case the growth is $ O(situations) $. However an implementor can also break the system with blocking or long execution times. \\
Especially two points could improve performance when evaluating predicates:

\begin{itemize}
\item Parallel evaluation of predicates.
\item Implementation of an algorithm to only check relevant entity types for a situation.
\end{itemize}

\paragraph{Communication}
The implementation of OconCommunication is a very large part of the framework. As per design goal 1 (see scope in section \ref{scope}) our framework should be adaptable. Therefore OconCommunication is designed in a way so that developers can make their own implementation and dependency inject it into Ocon. The goal is achieved, but the solution could have been better. As it is now, developers have to change a very large part of the system. Reflecting on the implementation, a better design could have been achieved by decomposing IOconCom and OconTcpCom into following three components.



\begin{enumerate}
\item An interface for communication containing a send, listen, broadcast and discover method.
\item An interface for serialization containing serialize and deserialize.
\item A class using concrete implementations of the two interfaces above to achieve the functionality in the current OconTcpCom.
\end{enumerate}

This would have resulted in a much lower coupled design making it much easier for developers to implement their own serialization or communication strategy.

Above reflections on communication, performance and serialization are obvious topics for future work to improve Ocon.

\paragraph{Proof of concept}
When implementing Ocon we experienced a framework that is easy to use and has the flexibility needed for our proof of concept. Especially the OconTcpCom implementation increased ease of use in our case, but imaging that another communication than tcp was needed, Ocon would function exactly the same given that this communication implementation was done correctly. Apart from that ease of use has been good in the simplicity of initializing the different Ocon components.

With that said one implementation by the developers of the framework only gives a limited view on its real-world usefulness





\end{document}