\documentclass[]{report}

\usepackage{float}
\usepackage[autostyle]{csquotes} 
\usepackage{parskip}
\usepackage{graphicx}
\usepackage{placeins}
\usepackage{wrapfig}
\usepackage{epigraph}
\usepackage[T1]{fontenc}
\usepackage{titlesec, blindtext, color}
\definecolor{gray75}{gray}{0.75}
\newcommand{\hsp}{\hspace{20pt}}
\titleformat{\chapter}[hang]{\Huge\bfseries}{\thechapter\hsp\textcolor{gray75}{|}\hsp}{0pt}{\Huge\bfseries}
\makeatletter
\newif\if@right
\def\shadequote{\@righttrue\shadequote@i}
\def\shadequote@i{\begin{snugshade}\begin{quote}\openquote}
\def\endshadequote{%
  \if@right\hfill\fi\closequote\end{quote}\end{snugshade}}
\@namedef{shadequote*}{\@rightfalse\shadequote@i}
\@namedef{endshadequote*}{\endshadequote}
\makeatother


\newcommand\todo[1]{\textcolor{red}{TODO: }#1\PackageWarning{TODO:}{TODO tag!!}}



\usepackage{listings}
\usepackage{color}
\usepackage{courier}

\definecolor{dkgreen}{rgb}{0,0.6,0}
\definecolor{gray}{rgb}{0.5,0.5,0.5}
\definecolor{mauve}{rgb}{0.58,0,0.82}

\lstset{frame=tb,
  language=[Sharp]C,
  aboveskip=3mm,
  belowskip=3mm,
  showstringspaces=false,
  columns=flexible,
  basicstyle={\small\ttfamily},
  numbers=none,
  numberstyle=\tiny\color{gray},
  keywordstyle=\color{blue},
  commentstyle=\color{dkgreen},
  stringstyle=\color{mauve},
  breaklines=true,
  breakatwhitespace=true
  tabsize=3
}




\begin{document}

\graphicspath{{img/}{../img/}}
\begin{titlepage}
\begin{center}

\vspace{2cm}

\rule{\linewidth}{0.6mm}

\textsc{\LARGE context-aware framework}
\vspace{0.2cm}
\rule{\linewidth}{0.4mm}

Jacob B. \textsc{Cholewa}
\\ \& \\
Mathias K. \textsc{Pedersen}

\vfill

\large \today
\end{center}


\end{titlepage}

\begin{abstract}
\end{abstract}


\chapter{Project description}
A Context-aware system is able to adapt its behaviour to the surroundings it is in. In order to act upon its environment, the system will need sensor-input. Various sources of input can be used to determine the actions of the system.\\

In this project we want to mainly focus on implementing a context-awareness framework, making it easy for systems to actuate on sensor events. As a proof of concept we will explore replacing the physical SCRUM board, which is often a whiteboard and post-its, with an IT-solution which is not confined to a personal computer, but has the same presence. The digital SCRUM board will be context-aware to the extent that it can recognize different SCRUM activities, like sprint meeting or one-on-one, and automatically change its graphical interface.


\chapter{Scope of this project}

As per our Project description there are two artefacts to be made; The context awareness framework and a proof of concept implementation in form of a Context-aware Scrum board.

\paragraph{The Framework.} 
Our project description dictates that the main piece of work will be our framework, and so implementation of this framework will have the broadest scope.\\

\begin{itemize}
\item[\textbf{Goal 1 }] \textit{Our aim is to implement a framework that is adaptable and easy to use.}
\end{itemize}


\paragraph{The Scrum board.} 
For us to prove usability of the framework developed we will implement it for a context-aware Scrum board. The function of this artefact is solely to test the usefulness of the context-awareness framework.\\

\begin{itemize}
\item[\textbf{Goal 2 }] \textit{In trying to prove Goal 1, implement a stub Scrum board utilizing the framework}
\end{itemize}

It is not in our scope to implement a functional or useful Scrum board.

\chapter{Background Research}
\section{Context awareness}
\label{sec:Context awareness}
To be able to make a context-aware framework, we first had to investigate what context and context-awareness is in a computer science perspective.

Context-awareness is a term associated with Ubiquitous computing. Ubiquitous computing, ubicomp, was coined in the early nineties by Mark Weiser whose vision was to make technology that could seamlessly assist in everyday tasks. Weisers research-unit at Xerox PARC developed some of the first mobile devices, and the development of ubi computing clearly reflects in todays technology boom of smart phones and tablets.

For a system to seamlessly interact with a human the system must acquire knowledge to the current situation or \textit{context}. Many of the publications on the subject describes context different, but the one description fitting best our understanding was coined by Dey and Abowd whom described context as:

\blockquote{\textit{Any information that can be used to characterize the situation of an entity. An entity is a person, place or object that is considered relevant to the interaction between a user and an application, including the user and application themselves.}} \cite{Dey and Abowd (2000)} 

When humans interact, they can interpret the ongoing situation from implicitly understanding body language, tone of voice, the surrounding environment as well as maybe having a relation to the person or knowing about past and future events. These \textit{context informations} is essential factors in effective communication. By increasing the amount of context information available to computers, human-machine interaction can be improved.\\

% Holy grail is to understand and perform human intent
\todo{We strive to understand human intent, with sensors. We might not catch it all, but that's not because it's a machine, this can happen in all kinds of interactions; culture differences... Something on this}


Much research have been done in increasing the available amount of context-information, but also heightening the quality and correctness of the information is in focus. As the amount of context increases, the context-aware application becomes able to take actions without explicit user input. When doing so, action can be taken on wrong or typically incomplete snaps of context. 


\blockquote{\textit{Intelligibility and control are important user concerns in context-aware applications. They allow a user to understand why an application is behaving a certain way, and to change its behaviour.}} \cite{Dey and Newberger (2009)}
\todo Write more about context accuracy


\section{Context aware frameworks}
% Frameworks have been developed to accomadate this
To support application developers a number of context-frameworks have been developed. When looking at the frameworks mainly two approaches were used: Blackboard and Widget-based architecture.

The blackboard approach is a centralized solution. Sensors and application are connected to the blackboard and when ever a new sensor state is available a post-it, an entry to the database, is put on the blackboard. The application can at will look through the blackboard and search for context it might find relevant. The blackboard abstracts away the sensor allowing the client to focus on only the context information. 

\begin{figure}
\centering
\includegraphics[width=\linewidth]{blackboard-widget.png}
\caption{Blackboard and widget-based approach}
\label{fig:blackboard-widget}
\end{figure}

The widget approach is an object-oriented distributed solution. Sensors encapsulated by widgets are available for application subscription. The solution is event based and applications are notified whenever a change to the sensor is occurring. This solution is object-oriented as the context is modelled with objects sent from sensor to application. This time and spare coupled solution stands in contrast to the blackboard approach which is a database and is therefore time- and space uncoupled.

The solution differs a lot in the way the model context, but the main differers is in the way of delivering context from sensor to application. The solutions stands in great contrast when looking at space and time coupling. To briefly describe the theory, space coupling is weather or not the sender knows who the receiver of a message is. Time coupling is if the given message is only available in real-time. 

Where the blackboard at any time offers applications to go though it's context database, it does not offer notifying the application, as the blackboard is space uncoupled, and does not know about the application. The widget solution offers live updated only when they happens and only to the applications subscribing for the update.

Both solutions are very useful in different applications.

\todo Should the two examples stay?\\
For example a hospital system where you want to use the context framework to track patients, the blackboard solution seams to meet requirements best as you can, when needed, look up a patients whereabouts in the hospital.

For a home automation system using sensors, sensor input is only interesting the moment it happens, and only to actuator whom it concern. When a person enters the room the light should go on instantly, only in the room where the person entered and only at the given time. 


\chapter{Design}
This chapter describes the different design choices we have though about in developing the framework.

To establish a vocabulary:
	Situation - A physical situation eg. A person sitting down

	Context - Context is knowing the situation and being able to react and be accordingly

	Context information - Every information that aggregates a context

\section{Grand Architecture}
When looking at the blackboard and widget based methods, we have decided that we want to include a little of both.

We want to do a centralized system where clients can register a situation predicate. Our implementation, the OCon framework, will be event-based so when ever the state of a clients predicate changes, an event will be fired to notify the client. 

This combines the blackboards centralization and abstraction with the widgets time and space coupling. This will result in a solution where its transparent for the developer which sensor is actually given the input, like the blackboard, but also having the space and time coupling from the widget based method.

With this OCon will be an If-this-then-that solution where the client can use sensor input for control without the client developers having to put much though into using and managing sensors. This will be realized by three different components. A client, a central and a widget. 

\begin{figure}
\centering
\includegraphics[width=200px]{grandarchitecture.png}
\caption{Grand architecture of OCon}
\label{fig:composite}
\end{figure}

\todo Please QA

\section{Encapsulation of Context-information}
Object-oriented encapsulation of context can be done in a variety of ways.

We have been looking at two different methods. Using a composite pattern and modelling with properties.

Using a composite pattern enforced relations between entities, entities being locations, persons, things or other real world objects.

Lets say we want to model a location with some rooms with some persons with different items or actions. Using the composite modelling we can have a location object containing room objects containing person objects containing item objects. We'll then have a relationship between the objects compositing the context being that a group of students is in room 3A04 at ITU all having phones in their pockets and all sitting down. This is a fairly complex, but a very extensible and flexible way to model situation (See fig \ref{fig:composite}). The downside to this method is that you can easily have an overflow of entities, and that it can be very complex and computation heavy to check against a predicate.

The other method we are considering is more simple. Having a set of entities that we wish to track, eg persons. All context information relating to a person will be properties to the object. So to use the previous example, a person entity would then have a location property, being ITU room 3A04, a phone property begin true and a sitting property being true. This model is more easy to make, but it limits the developer to know very precisely what information is needed and should be tracked. The previous model had the advantage of being very flexible allowing more complex relations and situations. 

Both methods have pros and cons being how dynamic they are, how easily they can be implemented, and what performance they will have and that will be the parameters we'll look at when implementing OCon 

\begin{figure}
\centering
\includegraphics[width=200px]{composite.png}
\caption{Diagram illustrating composite modeling}
\label{fig:composite}
\end{figure}


\section{Central}

Then central is the central competent and will facilitate clients adding predicates for the central to track. When an entity change is received relevant to the predicate, the predicate will be checked and if its state changes, the owning client will be notified.

The definition of predicate is taken from the .NET platform. A predicate is a delegate returning a boolean. In this way .NET developers will be familiar with the meaning and definition of a predicate.

A important feature to be included with our central is the ability to be discovered by client and widgets. The central will be able to broadcast itself to the chosen distribution channel and to widget and client will then be able to discover and connect to the central. This feature will enforce a more dynamic usage and behaviour of OCon.

\todo Please QA
 

\section{Widget}

In OCon widgets will translate sensor input to entities along with facilitating the communication to the central.

This approach have been chosen to make the system easily distributed as a small, maybe embedded, component can facilitate as widget translating the sensors input to entities before sending them to the central.

This approach will also reduce network load as the widget will only send updates to the central when changes occur. This stands in contrast to sending all sensor input to the central and making it process the data itself which could be useful in some cases.

Latests would also put more computational pressure on the central making it hard to scale to larger systems.


\section{Client}

The client will be the entry point into OCon facilitating developers to add situations to the central and receive updates whenever an update to those situations occurred.

The client will be able to automaticly discover and use the central. 

\todo What do we do with this chapter?


\section{Communication}

Initially we decided that making the OCon distributed would be out scoped, but as the project progressed it was decided to bring it into scope. Clients and widgets will be distributed from the central. The reason for doing so is that our vision of having a centralized system distributing situation from sensors to clients is not very useful if not distributed.

For the OCon two different communication protocols need to be implemented.\\

\begin{itemize}
	\item A protocol for establishing link between peers
	\item A protocol for send text/json messages for subscriptions, subscription events and, sensor events \\
\end{itemize}

We do not wish to bind our users to any concrete communication protocols. Therefore the communication will be interfaces so a concrete implementation can be dependency injected into the OCon.

OCon will contain a default implementation build using the TCP/IP layer. For serialization we have chosen to use json. These decisions have been made so that developers are not bound to the .NET platform and can make clients and widgets in other languages like Java, or C directly on a micro controllers like Arduino.

For peer discovery we have chosen to use IP multicast. The central will broadcast itself for peers to discover. The peer, clients and widgets, will be listening on the multicast endpoint and will invoke an event when a central is discovered. 

\chapter{Implementation}


\section{Overview}

\begin{center}
\includegraphics[scale=0.2]{ComponentDiagram.png}
\end{center}


\paragraph{The Client} is an observer interested in sensor data. It subscribes to a situation on the ContextCentral.

\paragraph{The Widget} wraps a sensor and translates the raw data to entities which are sent to the ContextCentral.

\paragraph{The ContextCentral} is the central for entity information and situations. When an entity is received from a widget it is saved to the central and all situations are checked and subscribers notified if their statuses changed due to the update.



\section{Encapsulation of Context}
Implementation choices in encapsulating context

Entities are generalized by the type IEntity, specifying general information for all entities.


\begin{center}
\includegraphics[scale=0.15]{ContextClassDiagram.png}
\end{center}

\section{Situation predicate evaluation}

\section{Widget}

The widget's purpose is to track entities and keep them updated in the central. The widget developer should translate sensor input to entities and then the OCon widget will facilitate tracking the entity and sending it to the central. (See figure \ref{seqwidget})

When the widget is notified about an entity from the sensor first it's checked weather or not the entity is already tracked by the widget. If not tracked it will be allocated a new GUID before it is sent to the central. The GUID is a unique ID used for distributed systems and allows OCon to distinguish between adding the entity as a new entity or updating an entity already know to the framework.

\begin{figure}
\centering
\includegraphics[width=\linewidth]{sequencediagram-widget.png}
\caption{Information flow from Widget to central}
\label{fig:seqwidget}
\end{figure}


\section{Central}

\section{Client}

The client's purpose it to sent predicates to the central for tracking. When a situation update is sent from the central the client must be able to notify the parent application about the situation update. (See figure \ref{seqclient})

\begin{figure}
\centering
\includegraphics[width=\linewidth]{clientsequencediagram.png}
\caption{Information flow between client and central}
\label{fig:seqclient}
\end{figure}

When the central is discovered the client will sent its situations to the central. The central will compute the situation state and sent it back. Whenever the situation state changes the central will sent an update to the client containing the situations new state. The client will then fire an event which the application can listen and react to.

\section{Communication}



See figure \ref{fig:widgetComHelper} and \ref{fig:clientComHelper} for a detailed view of the communication between client and central and widget and central

\begin{figure}
\centering
\includegraphics[width=\linewidth]{comHelperSequence-widget.png}
\caption{Widget to Central sequence diagram}
\label{fig:widgetComHelper}
\end{figure}

\begin{figure}
\centering
\includegraphics[width=\linewidth]{comHelperSequence-client.png}
\caption{Client to Central sequence diagram}
\label{fig:clientComHelper}
\end{figure}



\chapter{Proof of concept}

\blockquote{
We can only evaluate the user experience afforded by the toolkit and its features by building applications that use it, and then evaluating them. While the toolkit itself can be evaluated on its technical criteria, the aspects of it that are designed to support a particular user experience can only be evaluated in the context of use and thus must be evaluated indirectly - through applications built with the toolkit. \cite{Infrastructure (2003)} \\}

In previous chapters we evaluated Ocon on it's technical criteria by the design and choices made. In this chapter we will answer to our \textit{Goal 2} by implementing Ocon.

\section{Motivation}

Proof of concept will be a digital context-aware Scrumboard, and in this section we will briefly describe our motivation for this choice.

%Scrum is a popular agile process of developing software. Originally thought as using a whiteboard for scheduling development tasks, it has been hard for to make use of Scrum in distributed teams, or 



%Scrum is a popular agile process for developing software. In modern Scrum the task board is center of activity planning. It can be a whiteboard where tasks are coordinated, estimated and assigned. For more on Scrum see \textit{Scrum.org}.
%Center of focus is the whiteboard where tasks are coordinated, estimated and assigned. For more on Scrum see \textit{scrum.org}.

Scrum is a popular agile process for developing software. It focuses on rigid disciplines with timeboxing. Our motivation lies as part of the discipline of sprinting, and more precisely on the Scrumboard which has been adapted as a means to articulate the tasks in focus. This articulation is an important factor in teamwork which under the term articulation work describes the effort a team puts into communication.

The classic physique of the Scrumboard is a whiteboard with post-its, but two reasons have been the driving force behind development of digital Scrumboards.

\begin{itemize}
\item The whiteboard is analog, and there is labor in digitalizing it, for example for a Scrummaster to calculate burndowns or velocity.
\item Distributed Scrum teams have become more frequent with off-shoring and they need a digital solution for organizing tasks.
\end{itemize}

Many tools have sprung up motivated of these factors\footnote{Confluence, Scrumwise, Team Foundation, Trello}. These tools are not on par with the whiteboard when it comes to decreasing articulation work, because they lack the physical presence that the whiteboard has with a team. They do however solve the problems of distribution and digitalization.

\begin{itemize}
\item[\textbf{Motivation 1}] \textit{Popular digital Scrum tools are intended to be used by individuals. This increases articulation work}
\end{itemize}

Expanding on this is the motivation for context awareness as described in section \ref{sec:Context awareness} on page \pageref{sec:Context awareness}.

\begin{itemize}
\item[\textbf{Motivation 2}] \textit{Context awareness can improve the interaction between technology and humans}
\end{itemize}



Going forward in this chapter we will look to implement an idea of how these motivations could be satisfied by means of a digital Scrumboard and Ocon.


%We ourselves have used Scrum for several projects but never with the pleasure of a physical Scrumboard. We never had a room allocated and it simply wasn't possible to maintain a physical Scrumboard in a new space every day.


%A lot of tools for this problem have sprung up\footnote{Confluence, Scrumwise, Team Foundation, Trello} and made Scrum possible in this scenario, and even for a fully distributed team who would not be gathered physically in the first place.




%\begin{itemize}
%\item[] \Large{There is no solution today that digitalizes the whiteboard while maintaining it's presence}
%\end{itemize}

%\normalsize
%\vspace{0.2cm}

%This project does not cover implementation of a solution to these problems, but a some good ideas on it




\section{Context awareness}

The Scrumboard's purpose as a boundary object is to display information relevant to the team. The relevant information for the team depends on the situation, 
We have chosen three situations in Scrum where we believe the Scrumboard can benefit the team in having Context awareness

\begin{itemize}
\item Overview
\item Closeup
\item Standup
\end{itemize}

There are two situations we deem important which we want the Scrumboard to act upon:

\begin{itemize}
\item The standup meeting: When more than one person is standing in front of the Board
\item An individual closeup with the board
\end{itemize}

Actuation on these situations is purely graphical. There will be change in the graphical interface according to which information the interactors are interested in given the situation.

\begin{itemize}
\item Standup
\item Closeup
\item Overview. This view is our default for when none of the other contexts are true
\end{itemize}


\todo{more explanation of the context involved and how we'll represent it with OCon...}



\section{Implementation and environment}

The Context-aware Scrumboard consists of three parts: The KinectEntitySensor, The Centralization and the Scrumboard. These parts are distributed and communicate with the OconTcpCom implementation over LAN from each of their own hardware nodes.

\begin{figure}[H]
\includegraphics[width=\linewidth]{./ProofOfConceptDeployment.png}
\caption{Proof of concept deployment}
\label{fig:ProofofConceptDeployment}
\end{figure}

\subsection{The KinectEntitySensor}
The KinectEntitySensor's task is to gather Context Information and send it to the centralization. Ocon has been implemented to help with that. This part of the Context-aware Scrumboard implementation involves transferring information to the OconCentral, and it is for that purpose the OconWidget has been implemented.

The widget's task is to translate sensor data into IEntity implementation types and pass them to the OconWidget which will handle transfering it to the central.
\begin{figure}[H]
\begin{lstlisting}
//Instantiate a logging instance
var log = Console.Out;
//for file logging: new StreamWriter("/file/path/here");

//Instantiate an IOconCom implementation
var com = new OconTcpCom(log);

//Instantiate widget
var widget = new OconWidget(com, log);

//Start searching for a central with the given IOconCom implementation
widget.StartDiscovery();

//Pass an entity to be added/updated at the central
var entity = new Person() { Name = "Mat", Present = true };
widget.Notify(entity);
\end{lstlisting}
\caption{Widget usage from TestWidget.Program}
\label{code:OconWidget}
\end{figure}



\subsection{The centralization}
The central
\begin{figure}[H]
\begin{lstlisting}
//Instantiate a logging instance
var log = Console.Out;

//Instantiate an IOconCom implementation
var tcpCom = new OconTcpCom(log);

//Instantiate the context filter
var oconFilter = new OconContextFilter(log);

//Instantiate situations with names and predicates
var closeupSituation = new Situation("Closeup", e => e.OfType<Person>().Count(p => p.Present) == 1);
var standupSituation = new Situation("Standup", e => e.OfType<Person>().Count(p => p.Present) == 2);

//Add the situations to the filter
oconFilter.AddSituation(closeupSituation, standupSituation);

//Instantiate the central
var central = new OconCentral(oconFilter, tcpCom, log);

//Initialize the central
central.Initialize();
\end{lstlisting}
\caption{Central usage from TestCentral.Program}
\label{code:OconCentral}
\end{figure}

\subsection{The Scrumboard}

\begin{figure}[H]
\begin{lstlisting}
//Choose a logging instance if any
var log = Console.Out;

//Instantiate a network helper. Here passing the logging target
//alternatively instantiate as new TcpHelper(); if no logging is needed
var comHelper = new OconTcpCom(log);

//Instantiate the client with communication, log, and params of situation names strings
var oconClient = new OconClient(comHelper, log, StandupSituationString, CloseupSituationString);

//Subscribe a delegate to be run when a situation change event is fired
oconClient.SituationStateChangedEvent += (sender, args) => UpdatePicture(args.SituationName, args.State);
\end{lstlisting}
\caption{Client usage from OconScrumBoard.MainViewModel}
\label{code:OconClient}
\end{figure}


\todo{This solution with the elements from blackboard makes development very uncoupled. The blackboard requires a consensus and so Widget/sensor and client can be implemented independently}

\chapter{Conclusion}

\begin{thebibliography}{9}

\bibitem{Dey and Abowd (2000)}
  Dey, A. K., and Abowd, G.D.
  \emph{Towards a better understanding of context and context-awareness. Workshop on the What, Who, Where, When and How Context-awareness, afflicted with the 2000 ACM Conference on Human Factors in Computer Systems},
  2000.
  
\bibitem{Dey and Newberger (2009)}
  	Anind K. Dey and Alan Newberger
  	\emph{Support for Context-Aware Intelligibility and Control}, 2009
  
  
\bibitem{Infrastructure (2003)}
  W. Keith Edwards, Victoria Bellotti, Anind K. Dey,
  Mark W. Newman.
  \emph{Stuck in the Middle: The Challenges of
  User-Centered Design and Evaluation for Infrastructure},
  2003.
  
\bibitem{Context-aware computing (2010)}
	Anind K. Dey
	\emph{Context-aware computing} in Ubiquitous Computing Fundamentals, 2010, 321-352

\end{thebibliography}
\chapter{Appendixes}

\end{document}
